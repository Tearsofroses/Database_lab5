% ==========================================
\section{Conclusion}
% ==========================================

This laboratory exercise demonstrated the implementation of various database constraints and programming constructs in MySQL. The key concepts covered include:

\begin{itemize}
    \item \textbf{Views:} Created multiple views to simplify complex queries and provide different perspectives on the data, including employee department views, project views, and supervisor/supervisee relationships.
    
    \item \textbf{Triggers:} Implemented triggers for enforcing business rules (salary limits, supervision rules, project management constraints), maintaining derived attributes, logging changes, and preventing unwanted operations.
    
    \item \textbf{Stored Functions:} Created user-defined functions to encapsulate reusable logic, such as counting employee project assignments.
    
    \item \textbf{Stored Procedures:} Developed procedures using cursors and control flow statements to process and display data with formatted output.
    
    \item \textbf{Constraint Types:} Explored different constraint implementation techniques including:
    \begin{itemize}
        \item Domain/attribute constraints (CHECK constraints)
        \item Table-level constraints
        \item Triggers for complex business rules
        \item INSTEAD OF triggers for view operations
    \end{itemize}
\end{itemize}

The exercises also highlighted the differences between various database management systems (MySQL, SQL Server, PostgreSQL, Oracle) in their support for features like INSTEAD OF triggers and assertion constraints.

Understanding these database programming concepts is essential for:
\begin{itemize}
    \item Maintaining data integrity
    \item Enforcing business rules at the database level
    \item Creating efficient and reusable database code
    \item Designing robust database applications
\end{itemize}
