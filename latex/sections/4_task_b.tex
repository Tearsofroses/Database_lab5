% ==========================================
\section{Task (b) - Num\_of\_Emp Derived Attribute}
% ==========================================

\textbf{Exercise:} Alter table Department to add the attribute Num\_of\_Emp that stores the number of employees working for each department. This attribute is a derived attribute from Employee.DNO and its value must be automatically calculated.

\textbf{Solution:}

\begin{lstlisting}
-- Add the column
ALTER TABLE DEPARTMENT ADD COLUMN Num_of_Emp INT DEFAULT 0;

-- Initialize the column with current counts
UPDATE DEPARTMENT d
SET Num_of_Emp = (SELECT COUNT(*) FROM EMPLOYEE e WHERE e.Dno = d.Dnumber);

-- Trigger for INSERT
DROP TRIGGER IF EXISTS trg_update_num_emp_insert;
DELIMITER //
CREATE TRIGGER trg_update_num_emp_insert
AFTER INSERT ON EMPLOYEE
FOR EACH ROW
BEGIN
    IF NEW.Dno IS NOT NULL THEN
        UPDATE DEPARTMENT 
        SET Num_of_Emp = Num_of_Emp + 1 
        WHERE Dnumber = NEW.Dno;
    END IF;
END //
DELIMITER ;

-- Trigger for DELETE
DROP TRIGGER IF EXISTS trg_update_num_emp_delete;
DELIMITER //
CREATE TRIGGER trg_update_num_emp_delete
AFTER DELETE ON EMPLOYEE
FOR EACH ROW
BEGIN
    IF OLD.Dno IS NOT NULL THEN
        UPDATE DEPARTMENT 
        SET Num_of_Emp = Num_of_Emp - 1 
        WHERE Dnumber = OLD.Dno;
    END IF;
END //
DELIMITER ;

-- Trigger for UPDATE
DROP TRIGGER IF EXISTS trg_update_num_emp_update;
DELIMITER //
CREATE TRIGGER trg_update_num_emp_update
AFTER UPDATE ON EMPLOYEE
FOR EACH ROW
BEGIN
    IF OLD.Dno IS NOT NULL AND (NEW.Dno IS NULL OR OLD.Dno != NEW.Dno) THEN
        UPDATE DEPARTMENT SET Num_of_Emp = Num_of_Emp - 1 WHERE Dnumber = OLD.Dno;
    END IF;
    IF NEW.Dno IS NOT NULL AND (OLD.Dno IS NULL OR OLD.Dno != NEW.Dno) THEN
        UPDATE DEPARTMENT SET Num_of_Emp = Num_of_Emp + 1 WHERE Dnumber = NEW.Dno;
    END IF;
END //
DELIMITER ;
\end{lstlisting}

\subsubsection*{Test Validation}
\begin{lstlisting}
-- Check current department counts
SELECT Dnumber, Dname, Num_of_Emp FROM DEPARTMENT;
\end{lstlisting}

\textbf{Expected Output:}
\begin{center}
\begin{tabular}{|c|l|c|}
\hline
\textbf{Dnumber} & \textbf{Dname} & \textbf{Num\_of\_Emp} \\
\hline
1 & Headquarters & 1 \\
4 & Administration & 2 \\
5 & Research & 5 \\
\hline
\end{tabular}
\end{center}

\begin{lstlisting}
-- Test INSERT: Add new employee to Dept 5
INSERT INTO EMPLOYEE VALUES ('New', 'N', 'Emp', '999999999', '1990-01-01', 
                             '123 St', 'M', 25000, '333445555', 5);
SELECT Dnumber, Num_of_Emp FROM DEPARTMENT WHERE Dnumber = 5;
-- Result: Num_of_Emp = 6 (incremented from 5)

-- Test DELETE: Remove the employee
DELETE FROM EMPLOYEE WHERE Ssn = '999999999';
SELECT Dnumber, Num_of_Emp FROM DEPARTMENT WHERE Dnumber = 5;
-- Result: Num_of_Emp = 5 (decremented back)
\end{lstlisting}
