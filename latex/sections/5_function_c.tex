% ==========================================
\section{Function (c) - Get Total Projects}
% ==========================================

\textbf{Exercise:} Write a function that returns the total number of projects when given an employee's ID.
\begin{itemize}
    \item \textbf{Input:} employee ID
    \item \textbf{Output:} total number of projects
\end{itemize}

\textbf{Solution:}

\begin{lstlisting}
DROP FUNCTION IF EXISTS GetTotalProjectsForEmployee;
DELIMITER //
CREATE FUNCTION GetTotalProjectsForEmployee(emp_ssn CHAR(9))
RETURNS INT
DETERMINISTIC
READS SQL DATA
BEGIN
    DECLARE total_projects INT;
    
    SELECT COUNT(*) INTO total_projects
    FROM WORKS_ON
    WHERE Essn = emp_ssn;
    
    RETURN total_projects;
END //
DELIMITER ;

-- Example usage:
SELECT GetTotalProjectsForEmployee('123456789') AS Total_Projects;
SELECT GetTotalProjectsForEmployee('333445555') AS Total_Projects;
\end{lstlisting}

\textbf{Explanation:} This function takes an employee SSN as input and returns the count of projects that employee works on from the WORKS\_ON table.

\subsubsection*{Test Validation}
\begin{lstlisting}
-- Call function for employee '123456789'
SELECT GetTotalProjectsForEmployee('123456789') AS Total_Projects;
\end{lstlisting}

\textbf{Expected Output:}
\begin{center}
\includegraphics[width=0.4\textwidth]{Images/ex1/functions/fc1.png}
\end{center}

\begin{lstlisting}
-- List all employees with their project counts
SELECT Ssn, CONCAT(Fname, ' ', Lname) AS Name, 
       GetTotalProjectsForEmployee(Ssn) AS Projects
FROM EMPLOYEE ORDER BY Projects DESC;
\end{lstlisting}

\textbf{Expected Output:}
\begin{center}
\includegraphics[width=0.8\textwidth]{Images/ex1/functions/fc4.png}
\end{center}
